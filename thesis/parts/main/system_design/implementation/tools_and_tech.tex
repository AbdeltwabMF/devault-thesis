%%% Local Variables:
%%% mode: latex
%%% TeX-master: t
%%% End:

\subsection{Tools and Technologies}

\begin{longtable}{p{0.15\linewidth} | p{0.80\linewidth}}
  \caption{Tools and technologies used in this project}
  \label{tab:toolsAndTech}
  \\\toprule
  \centering
  Tool & justification
  \\\midrule
  JavaScript & {
    JavaScript, often abbreviated JS, is a programming language that is one of the core technologies of the World Wide Web, alongside HTML and CSS. As of 2022, 98\% of websites use JavaScript on the client side for web page behavior, often incorporating third-party libraries. All major web browsers have a dedicated JavaScript engine to execute the code on users' devices.
  }
  \\\hline
  Next.js & {
    Next.js is an open-source web development framework built on top of Node.js enabling React-based web applications functionalities such as server-side rendering and generating static websites. React documentation mentions Next.js among ``Recommended Toolchains'' advising it to developers as a solution when ``Building a server-rendered website with Node.js''. Where traditional React apps can only render their content in the client-side browser, Next.js extends this functionality to include applications rendered on the server-side.
  }
  \\\hline
  Solidity & {
    Solidity is an object-oriented programming language for implementing smart contracts on various blockchain platforms, most notably, Ethereum. It was developed by Christian Reitwiessner, Alex Beregszaszi, and several former Ethereum core contributors. Programs in Solidity run on Ethereum Virtual Machine.
  }
  \\\hline
  Ethers.js & {
    The ethers.js library aims to be a complete and compact library for interacting with the Ethereum Blockchain and its ecosystem. It was originally designed for use with ethers.io and has since expanded into a more general-purpose library.
  }
  \\\hline
  Metamask & {
    MetaMask is a software cryptocurrency wallet used to interact with the Ethereum blockchain. It allows users to access their Ethereum wallet through a browser extension or mobile app, which can then be used to interact with decentralized applications. MetaMask is developed by ConsenSys Software Inc., a blockchain software company focusing on Ethereum-based tools and infrastructure.
  }
  \\\hline
  Hardhat & {
    Hardhat is an Ethereum development environment. Compile your contracts and run them on a development network. Get Solidity stack traces, console.log and more.
  }
  \\\hline
  IPFS & {
    The InterPlanetary File System (IPFS) is a protocol and peer-to-peer network for storing and sharing data in a distributed file system. IPFS uses content-addressing to uniquely identify each file in a global name-space connecting all computing devices.
  }
  \\\hline
  Docker & {
    Docker is a set of platform as a service (PaaS) products that use OS-level virtualization to deliver software in packages called containers. We use docker for shipping and self-hosting the dApp.
  }
  \\\hline
  Ropsten & {
    Ropsten Ethereum (also known as ``Ethereum Testnet'') is an Ethereum test network that allows for blockchain development testing before deployment on Mainnet, the main Ethereum network. Testnet ethers are separate and distinct from actual ethers, and are never supposed to have any value. This allows application developers or Ethereum testers to experiment, without having to use real ethers or worrying about breaking the main Ethereum chain.
  }
  \\\bottomrule
\end{longtable}
