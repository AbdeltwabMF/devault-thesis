\section{Introduction}

The current standard for digital data storage is called cloud storage. With cloud storage, users looking to host data, applications, and websites on the internet are reliant on centralized providers like Amazon, Google, and Microsoft to provide storage services. This method of storage — for which user data is stored on the centralized server farms of cloud storage providers — is often cheaper, more scalable, and more readily accessible across geographic regions than the previous standard of storage on physical hardware. \\[-8pt]

Cloud service providers allow developers to launch their applications more quickly, without worrying about setting up and managing servers, but customers typically have limited options in terms of providers and functionality. The majority of cloud storage providers are subsidiaries of bonafide tech giants and dominate the cloud services market, accounting for about 70\% of the total market share as of 2021. \\[-8pt]

Despite their popularity and widespread use, many centralized cloud storage providers have been criticized for their tendency to force end users into inflexible and expensive cloud services and storage plans due to a lack of viable alternatives. Studies have shown that many developers settle for fixed amounts of hosting space that remain underutilized. This often results in hefty — and in many cases, unnecessary — premiums paid for cloud services. \\[-8pt]

That said, perhaps the biggest concern with centralized data storage models is that users are required to place trust in the central authority of the provider to keep their data safe, keep websites online, and not tamper with or censor the content that the centralized data providers host. In response, blockchain technology and decentralized networks have fostered a whole new methodology for digital storage: decentralized cloud storage. \\[-8pt]

In contrast to centralized, permissioned cloud providers, decentralized cloud storage providers leverage infrastructure that is designed to mitigate undue control or influence. These providers typically also utilize a permissionless structure that enables developers to employ their services with reduced restrictions. Conceptually similar to a decentralized blockchain, decentralized storage models draw their security from their widely distributed structure. This overall architecture can help make these systems more resistant to the hackers, attacks, and outages that have plagued large, centralized data centers. \\[-8pt]