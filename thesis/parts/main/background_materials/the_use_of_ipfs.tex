%%% Local Variables:
%%% mode: latex
%%% TeX-master: t
%%% End:

\section{Use of IPFS, The Decentralized File System}

\subsection{What is IPFS}

The Interplanetary File System (IPFS) is a bundle of subprotocols and a project-driven by Protocol Labs, IPFS aims to improve the web’s efficiency and to make the web more decentralized and resilient. \\[-8pt]

IPFS uses content-based addressing, where content is not addressed via a location but via its content. IPFS stores and addresses data with its deduplication properties, allowing efficient storage of data. It also can be used as a storage service complementing blockchains, enabling different applications on top of IPFS. \\[-8pt]

\subsection{How IPFS works}

IPFS is a peer-to-peer storage network. Content is accessible through peers located anywhere in the world, that might relay information, store it, or do both. IPFS knows how to find what you ask for using its content address rather than its location. \\[-8pt]

\noindent
There are three fundamental principles to understanding IPFS: \\

\begin{itemize}
\item Unique identification via content addressing.
\item Content linking via directed acyclic graphs (DAGs).
\item Content discovery via distributed hash tables (DHTs).
\end{itemize}

These three principles build upon each other to enable the IPFS ecosystem.