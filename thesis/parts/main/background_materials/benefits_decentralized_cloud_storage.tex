\section{Benefits Decentralized Cloud Storage}

\subsection{Encrypted}

The nodes in a decentralized storage system are unable to see or modify your files since all data uploaded to a decentralized storage network is encrypted by default. As a result, you have unrivaled security and privacy, ensuring that your information is safe. Because of data encryption, nobody can access it without its unique hash. You can store personal and sensitive information without having any fear.

\subsection{Secured}

Decentralized data storage systems, provide a high level of security. They split the data into smaller chunks, produce copies of the original data, and then use hashes or public-private keys to encrypt each portion independently. The entire procedure protects the data from malicious parties.


\subsection{Flexible Load Balancing}

To make the process more efficient, blockchain-based decentralized storage systems allow the host to cache frequently-used data. It relieves server load and reduces network traffic. This eliminates the need for hosts to access the server on a regular basis to retrieve information.

\subsection{Less Computer Power with Band Width}

Decentralized cloud storage encrypts data, breaks it up, and distributes it for storage on drives. It is operated by various organizations in a variety of locations, each with its own power supply and network connection, creating something much less wasteful. A decentralized file storage system reduces both hardware and software expenses. You also don’t require high-performance equipment to use it efficiently.  More significantly, a decentralized network may include millions, if not billions, of nodes. This significantly increases the amount of storage space accessible. Decentralized data storage does not need high power consumption to run on the system rather it uses less computer power with Bandwidth.

\subsection{No dedicated Servers for Storage}

Decentralized cloud storage represents a paradigm shift to content-centric approach from a location-centric. One cannot access the database in decentralised cloud storage by just identifying ‘where it is. Because data is distributed across a global network rather than being kept in a selected point, the principle of location becomes void in decentralised cloud storage.

Unlike centralized storage systems where a finite few data centers host your data, decentralized storage networks are composed of a series of nodes eager to host the data in a secure manner. It does not only offer a wider range of storage bandwidth, but it also reduces the overall storage cost, making it a cost-effective option.

\subsection{Fast}

It is commonplace to encounter network bottlenecks with centralized storage systems as the network traffic may sometimes overwhelm the servers. In a decentralized storage network, though, multiple copies of data are stored across various nodes. This eliminates the probability of network bottlenecks as you can access your data from a huge number of nodes, in a fast and secure manner.

Above were some of the advantages of decentralized cloud storage over traditional cloud storage which do not need any explanation. Seeing above advantages we can say that it can be future of cloud storage in coming years.