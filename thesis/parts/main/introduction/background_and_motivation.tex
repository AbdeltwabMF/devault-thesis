%%% Local Variables:
%%% mode: latex
%%% TeX-master: t
%%% End:

\section{Background and Motivation}

Cloud storage has effectively replaced the traditional model of physical hardware storage for developers building apps and websites, as well as individual consumers storing their data. However, the centralized providers that provide cloud storage services have fostered a system with serious drawbacks like high fees, low flexibility, and a lack of alternatives. That’s where blockchain networks are working to improve upon the legacy model, striving to provide equitable decentralized cloud storage solutions that can better align the incentives of users and providers.

The internet now is governed by \acrfull{http}. And It’s how you access websites, watch video’s, download files. There are some problems with it however, a lot of it stemming from the fact that the current model is largely \gls{centralized} and this version of the web called \gls{web2}.

\Gls{web2} is the World Wide Web based on the concepts of social media, where the user can create content, post it online, and engage with other user-generated content. But the upcoming issue was that they did not own this content or the revenue being generated by it. The company that provided the platform for sharing the content has the maximum ownership of the revenue generated by that content. This led to the centralization of the data and traffic influence.

\Gls{web3}, unlike \gls{web2}, has a \gls{decentralized} distributed system. That means that all the nodes on the system in \gls{web3} have equal control and access. One of the key features of \gls{web3} is that it implements \gls{smart contract} and Token using the \gls{blockchain} mechanism.